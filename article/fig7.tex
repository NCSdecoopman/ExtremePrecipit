% Options for packages loaded elsewhere
% Options for packages loaded elsewhere
\PassOptionsToPackage{unicode}{hyperref}
\PassOptionsToPackage{hyphens}{url}
\PassOptionsToPackage{dvipsnames,svgnames,x11names}{xcolor}
%
\documentclass[
  border=0pt]{standalone}
\usepackage{xcolor}
\usepackage{amsmath,amssymb}
\setcounter{secnumdepth}{-\maxdimen} % remove section numbering
\usepackage{iftex}
\ifPDFTeX
  \usepackage[T1]{fontenc}
  \usepackage[utf8]{inputenc}
  \usepackage{textcomp} % provide euro and other symbols
\else % if luatex or xetex
  \usepackage{unicode-math} % this also loads fontspec
  \defaultfontfeatures{Scale=MatchLowercase}
  \defaultfontfeatures[\rmfamily]{Ligatures=TeX,Scale=1}
\fi
\usepackage{lmodern}
\ifPDFTeX\else
  % xetex/luatex font selection
\fi
% Use upquote if available, for straight quotes in verbatim environments
\IfFileExists{upquote.sty}{\usepackage{upquote}}{}
\IfFileExists{microtype.sty}{% use microtype if available
  \usepackage[]{microtype}
  \UseMicrotypeSet[protrusion]{basicmath} % disable protrusion for tt fonts
}{}
\makeatletter
\@ifundefined{KOMAClassName}{% if non-KOMA class
  \IfFileExists{parskip.sty}{%
    \usepackage{parskip}
  }{% else
    \setlength{\parindent}{0pt}
    \setlength{\parskip}{6pt plus 2pt minus 1pt}}
}{% if KOMA class
  \KOMAoptions{parskip=half}}
\makeatother
% Make \paragraph and \subparagraph free-standing
\makeatletter
\ifx\paragraph\undefined\else
  \let\oldparagraph\paragraph
  \renewcommand{\paragraph}{
    \@ifstar
      \xxxParagraphStar
      \xxxParagraphNoStar
  }
  \newcommand{\xxxParagraphStar}[1]{\oldparagraph*{#1}\mbox{}}
  \newcommand{\xxxParagraphNoStar}[1]{\oldparagraph{#1}\mbox{}}
\fi
\ifx\subparagraph\undefined\else
  \let\oldsubparagraph\subparagraph
  \renewcommand{\subparagraph}{
    \@ifstar
      \xxxSubParagraphStar
      \xxxSubParagraphNoStar
  }
  \newcommand{\xxxSubParagraphStar}[1]{\oldsubparagraph*{#1}\mbox{}}
  \newcommand{\xxxSubParagraphNoStar}[1]{\oldsubparagraph{#1}\mbox{}}
\fi
\makeatother


\usepackage{longtable,booktabs,array}
\usepackage{calc} % for calculating minipage widths
% Correct order of tables after \paragraph or \subparagraph
\usepackage{etoolbox}
\makeatletter
\patchcmd\longtable{\par}{\if@noskipsec\mbox{}\fi\par}{}{}
\makeatother
% Allow footnotes in longtable head/foot
\IfFileExists{footnotehyper.sty}{\usepackage{footnotehyper}}{\usepackage{footnote}}
\makesavenoteenv{longtable}
\usepackage{graphicx}
\makeatletter
\newsavebox\pandoc@box
\newcommand*\pandocbounded[1]{% scales image to fit in text height/width
  \sbox\pandoc@box{#1}%
  \Gscale@div\@tempa{\textheight}{\dimexpr\ht\pandoc@box+\dp\pandoc@box\relax}%
  \Gscale@div\@tempb{\linewidth}{\wd\pandoc@box}%
  \ifdim\@tempb\p@<\@tempa\p@\let\@tempa\@tempb\fi% select the smaller of both
  \ifdim\@tempa\p@<\p@\scalebox{\@tempa}{\usebox\pandoc@box}%
  \else\usebox{\pandoc@box}%
  \fi%
}
% Set default figure placement to htbp
\def\fps@figure{htbp}
\makeatother





\setlength{\emergencystretch}{3em} % prevent overfull lines

\providecommand{\tightlist}{%
  \setlength{\itemsep}{0pt}\setlength{\parskip}{0pt}}



 


\usepackage{graphicx}
\usepackage{tabularx}
\usepackage{array}
\usepackage{makecell}
\usepackage{multirow}
\usepackage{caption}
\usepackage[export]{adjustbox}

\newcommand{\rJAN}{0.11}
\newcommand{\nJAN}{159}
\newcommand{\meJAN}{-0.28}

\newcommand{\rFEB}{0.45}
\newcommand{\nFEB}{177}
\newcommand{\meFEB}{-75.64}

\newcommand{\rMAR}{0.04}
\newcommand{\nMAR}{283}
\newcommand{\meMAR}{-181.97}

\newcommand{\rAPR}{0.18}
\newcommand{\nAPR}{186}
\newcommand{\meAPR}{-11.53}

\newcommand{\rMAY}{0.02}
\newcommand{\nMAY}{231}
\newcommand{\meMAY}{-51.26}

\newcommand{\rJUN}{0.08}
\newcommand{\nJUN}{316}
\newcommand{\meJUN}{-136.12}

\newcommand{\rJUL}{-0.00}
\newcommand{\nJUL}{324}
\newcommand{\meJUL}{-11.74}

\newcommand{\rAUG}{0.14}
\newcommand{\nAUG}{239}
\newcommand{\meAUG}{-16.56}

\newcommand{\rSEP}{-0.00}
\newcommand{\nSEP}{209}
\newcommand{\meSEP}{-24.61}

\newcommand{\rOCT}{0.07}
\newcommand{\nOCT}{218}
\newcommand{\meOCT}{-50.84}

\newcommand{\rNOV}{0.26}
\newcommand{\nNOV}{222}
\newcommand{\meNOV}{-1.88}

\newcommand{\rDEC}{0.26}
\newcommand{\nDEC}{212}
\newcommand{\meDEC}{-29.33}

\makeatletter
\@ifpackageloaded{caption}{}{\usepackage{caption}}
\AtBeginDocument{%
\ifdefined\contentsname
  \renewcommand*\contentsname{Table of contents}
\else
  \newcommand\contentsname{Table of contents}
\fi
\ifdefined\listfigurename
  \renewcommand*\listfigurename{List of Figures}
\else
  \newcommand\listfigurename{List of Figures}
\fi
\ifdefined\listtablename
  \renewcommand*\listtablename{List of Tables}
\else
  \newcommand\listtablename{List of Tables}
\fi
\ifdefined\figurename
  \renewcommand*\figurename{Figure}
\else
  \newcommand\figurename{Figure}
\fi
\ifdefined\tablename
  \renewcommand*\tablename{Table}
\else
  \newcommand\tablename{Table}
\fi
}
\@ifpackageloaded{float}{}{\usepackage{float}}
\floatstyle{ruled}
\@ifundefined{c@chapter}{\newfloat{codelisting}{h}{lop}}{\newfloat{codelisting}{h}{lop}[chapter]}
\floatname{codelisting}{Listing}
\newcommand*\listoflistings{\listof{codelisting}{List of Listings}}
\makeatother
\makeatletter
\makeatother
\makeatletter
\@ifpackageloaded{caption}{}{\usepackage{caption}}
\@ifpackageloaded{subcaption}{}{\usepackage{subcaption}}
\makeatother
\usepackage{bookmark}
\IfFileExists{xurl.sty}{\usepackage{xurl}}{} % add URL line breaks if available
\urlstyle{same}
\hypersetup{
  colorlinks=true,
  linkcolor={blue},
  filecolor={Maroon},
  citecolor={Blue},
  urlcolor={Blue},
  pdfcreator={LaTeX via pandoc}}


\author{}
\date{}
\begin{document}


\begingroup
\noindent
\begin{minipage}[t]{\paperwidth}
\centering
\setlength{\tabcolsep}{0pt}
\renewcommand{\arraystretch}{1}

\begin{tabularx}{\paperwidth}{
  @{}
  >{\centering\arraybackslash}X
  >{\centering\arraybackslash}X
  @{}
}

% ---------- LIGNE 1 ----------
\textbf{\small JAN} \small ($r=\rJAN$,\, n=\nJAN,\, ME=\meJAN\%) &
\textbf{\small FEB} \small ($r=\rFEB$,\, n=\nFEB,\, ME=\meFEB\%) \\[-1mm]

\includegraphics[width=\linewidth, trim=12 12 12 12, clip]{figures/trend_horaire_pluie_jan.pdf} &
\includegraphics[width=\linewidth, trim=12 12 12 12, clip]{figures/trend_horaire_pluie_fev.pdf} \\[1mm]

% ---------- LIGNE 2 ----------
\textbf{\small MAR} \small ($r=\rMAR$,\, n=\nMAR,\, ME=\meMAR\%) &
\textbf{\small APR} \small ($r=\rAPR$,\, n=\nAPR,\, ME=\meAPR\%) \\[-1mm]

\includegraphics[width=\linewidth, trim=12 12 12 12, clip]{figures/trend_horaire_pluie_mar.pdf} &
\includegraphics[width=\linewidth, trim=12 12 12 12, clip]{figures/trend_horaire_pluie_avr.pdf} \\[1mm]

% ---------- LIGNE 3 ----------
\textbf{\small MAY} \small ($r=\rMAY$,\, n=\nMAY,\, ME=\meMAY\%) &
\textbf{\small JUN} \small ($r=\rJUN$,\, n=\nJUN,\, ME=\meJUN\%) \\[-1mm]

\includegraphics[width=\linewidth, trim=12 12 12 12, clip]{figures/trend_horaire_pluie_mai.pdf} &
\includegraphics[width=\linewidth, trim=12 12 12 12, clip]{figures/trend_horaire_pluie_jui.pdf} \\[1mm]

% ---------- LIGNE 4 ----------
\textbf{\small JUL} \small ($r=\rJUL$,\, n=\nJUL,\, ME=\meJUL\%) &
\textbf{\small AUG} \small ($r=\rAUG$,\, n=\nAUG,\, ME=\meAUG\%) \\[-1mm]

\includegraphics[width=\linewidth, trim=12 12 12 12, clip]{figures/trend_horaire_pluie_juill.pdf} &
\includegraphics[width=\linewidth, trim=12 12 12 12, clip]{figures/trend_horaire_pluie_aou.pdf} \\[1mm]

% ---------- LIGNE 5 ----------
\textbf{\small SEP} \small ($r=\rSEP$,\, n=\nSEP,\, ME=\meSEP\%) &
\textbf{\small OCT} \small ($r=\rOCT$,\, n=\nOCT,\, ME=\meOCT\%) \\[-1mm]

\includegraphics[width=\linewidth, trim=12 12 12 12, clip]{figures/trend_horaire_pluie_sep.pdf} &
\includegraphics[width=\linewidth, trim=12 12 12 12, clip]{figures/trend_horaire_pluie_oct.pdf} \\[1mm]

% ---------- LIGNE 6 ----------
\textbf{\small NOV} \small ($r=\rNOV$,\, n=\nNOV,\, ME=\meNOV\%) &
\textbf{\small DEC} \small ($r=\rDEC$,\, n=\nDEC,\, ME=\meDEC\%) \\[-1mm]

\includegraphics[width=\linewidth, trim=12 12 12 12, clip]{figures/trend_horaire_pluie_nov.pdf} &
\includegraphics[width=\linewidth, trim=12 12 12 12, clip]{figures/trend_horaire_pluie_dec.pdf} \\[1mm]

% ---------- LÉGENDE ----------
\mbox{\vspace{-2mm}} & \begin{minipage}[c]{\linewidth} \centering \makebox[0.85\linewidth][c]{\small \%} \includegraphics[width=0.85\linewidth,keepaspectratio]{../outputs/maps/gev_z_T_p/horaire/compare_12/sat_90.0/legend_horiz_signif.pdf}\vspace{-3mm} \end{minipage} \\

\end{tabularx}

\vspace{0.75em}
\end{minipage}
\endgroup




\end{document}
